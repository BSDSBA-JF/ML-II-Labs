\documentclass{extarticle}
\sloppy

%%%%%%%%%%%%%%%%%%%%%%%%%%%%%%%%%%%%%%%%%%%%%%%%%%%%%%%%%%%%%%%%%%%%%%
% PACKAGES            																						  %
%%%%%%%%%%%%%%%%%%%%%%%%%%%%%%%%%%%%%%%%%%%%%%%%%%%%%%%%%%%%%%%%%%%%%
\usepackage[10pt]{extsizes}
\usepackage{amsfonts}
\usepackage{amsthm}
\usepackage{amssymb}
\usepackage[shortlabels]{enumitem}
\usepackage{microtype} 
\usepackage{amsmath}
\usepackage{mathtools}
\usepackage{commath}
\usepackage[margin=1in]{geometry}
\usepackage{float}
\usepackage{cancel}
\usepackage{amsmath, amsfonts, amssymb}


%%%%%%%%%%%%%%%%%%%%%%%%%%%%%%%%%%%%%%%%%%%%%%%%%%%%%%%%%%%%%%%%%%%%%%
% DOCUMENT START              																			           %
%%%%%%%%%%%%%%%%%%%%%%%%%%%%%%%%%%%%%%%%%%%%%%%%%%%%%%%%%%%%%%%%%%%%%%
\title{\vspace{-2em}Credit Byte}
\author{
  \emph{LT \#7} \\[0.5em]
  Mariel Concepcion, JF Viray \\ 
  Enzo de Villa, Lindsey Sebastian \\ 
  Bingbong Manglicmot
}

\begin{document}
\maketitle

\section{Executive Summary}
CreditByte is a mid-sized financial services company, which struggles with fraud since criminals bypass traditional rule-based systems. 
For example, many fraudulent transactions are small in amount, making them difficult to flag with just a simple threshold check. 
This study applies machine learning to improve detection with special attention to class imbalance between 320 fraudulent and 184,804 non-fraudulent samples.
By using \underline{\hspace{5cm}}, the optimized model improved the Fraud Capture Rate (FCR) from \underline{\hspace{5cm}} to \underline{\hspace{5cm}}, successfully identifying transactions worth \$\underline{\hspace{5cm}}. 
These findings highlight machine learning as a stronger defense against fraud for CreditByte.

\section{Introduction}
Our overall goal in the EDA is:
1. to determine why the old methods of thresholding and unusual transaction amounts do not work anymore, and 
2. to guide the choice of preprocessing techniques. 

To do this, we will examine look into the extent of class imbalance and look for patterns that distinguish fraudulent from non-fraudulent transactions.
\section{Methodology}

\section{Results and Discussion}

\section{Conclusion and Recommendation}

\section{References}


\end{document}